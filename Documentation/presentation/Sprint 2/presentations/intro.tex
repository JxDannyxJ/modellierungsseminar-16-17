\newcommand\tab[1][1cm]{\hspace*{#1}}
\graphicspath{{images/}{images/logos/}}
\begin{frame}{Unsere Aufgabe}
	\begin{center}
		\begin{itemize}
			\item Simulation des Festzuges in der Landshuter Innenstadt
			\item Menschen, Pferde und Kutschen sollen eingebunden werden
			\item Weitergabe der Ergebnisse
		\end{itemize}
	
\end{center}
	\underline{Simulations Tool:} $\underline{}$
	\begin{itemize}
		\item OpenVadere (Open Source Projekt) 
	\end{itemize}
\end{frame}

\begin{frame}{Sprint Ziele}
	\begin{center}
		\begin{itemize}
			\item US-2: Horse Modell:
			\begin{itemize}
				\item Unabhängiges Bewegungsmodell für das Pferd festlegen
			\end{itemize}
			\item US-7: Schnittstelle:
			\begin{itemize}
				\item Datenweitergabe an die Unity Gruppe
				\item Weitergabe der Rechenergebnisse an die CityGML Gruppe
			\end{itemize} 
			\item US-8: Landshuter Hochzeit (Teilszenario):
			\begin{itemize}
				\item Realität nah ein Teilszenario der Landshuter Hochzeit abbilden
			\end{itemize}
			\item US-5: Kutsche (Optional): 
				\begin{itemize}
				\item Erstellung einer Kutsche
			\end{itemize}
		\end{itemize}
	\end{center}
\end{frame}

\begin{frame}{Erreichte Ziele}
	\begin{center}
		\begin{itemize}
			\item \colorbox{green} {US-2: Horse Modell:}
			\begin{itemize}
				\item \colorbox{green} {Unabhängiges Bewegungsmodell für das Pferd fetlegen}
			\end{itemize}
			\item \colorbox{yellow} {US-7: Schnittstelle:}
			\begin{itemize}
				\item \colorbox{green} {Datenweitergabe an die Unity Gruppe}
				\item \colorbox{red}{Weitergabe der Rechenergebnis an die CityGML Gruppe}
			\end{itemize} 
		
			\item \colorbox{green} {US-8: Landshuter Hochzeit (Teilszenario):}
			\begin{itemize}
				\item \colorbox{green}{Realität nah ein Teilszenario der Landshuter Hochzeit abbilden}
			\end{itemize}
			\item \colorbox{green}{US-5: Kutsche (Optional):} 
			\begin{itemize}
				\item \colorbox{green}{Erstellung einer Kutsche}
			\end{itemize}
		\end{itemize}
	\end{center}
\end{frame}

\begin{frame}{Schwierigkeiten und Lösungen}
	\begin{itemize}
		\item Viele mögliche Ideen, wenig Kapazitäten
		 \begin{itemize}
			\item Festlegen fester User Stories, als Ziel 
			\item Leitung durch Scrum Master welche Task Priorität haben
		\end{itemize} 
		\item Verstreutes Team
			 \newline Treffen:
			 \begin{itemize}
			 	\item Donnerstags (Vorlesung)
			 	\item Sonntags(Online)
			 	\item Dienstags (Freiwillig HM)
			 \end{itemize} 
	\end{itemize}
\end{frame}

\begin{frame}{Ausblick}
	\begin{itemize}
		\item Komplette Simulation der Landshuter Hochzeit in viele Teilszenarien
		\item Weitergabe der Daten an die CityGML Gruppe
	\end{itemize}
\end{frame}
