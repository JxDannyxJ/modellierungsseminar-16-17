\newChapterSlide{Daniel Jadanec}{Gruppenmodell, Verteilungsparameter, Funktionen zur Dichteanalyse}{}{35}{0}
\begin{frame}{Gruppenmodell nach Moussa\"{i}d}
	\begin{itemize}
		\item up to 70\% of observed pedestrians in a commercial street are walking in group \cite{moussaid-2010b}
		\item groups composed of two to four members are the most frequent \cite{moussaid-2010b}
		\item group sizes are distributed according to a Poisson distribution \cite{jamesj-1953}
	\end{itemize}
\end{frame}

\begin{frame}{Gruppenmodell nach Moussa\"{i}d}
	\begin{minipage}{0.50\textwidth}
		\includegraphics[width=\textwidth]{pres_pics/Groupsize.png}
		\captionof{figure}{Figure 2 in \cite{moussaid-2010b}}
	\end{minipage} \hfill
	\begin{minipage}{0.48\textwidth}
	\includegraphics[width=\textwidth]{pres_pics/groups.png}
	\captionof{figure}{Figure 3 in \cite{moussaid-2010b}}
\end{minipage}
\end{frame}
\begin{frame}{Implementierung}
	\begin{figure}
		\includegraphics[width=0.6\textwidth]{pres_pics/group_problem.png}\\
		$\Rightarrow$ nicht realisierbar mit Außenpersonenströmen
	\end{figure}
\end{frame}
\begin{frame}{Lösungsidee I}
	\begin{figure}
		\includegraphics[width=0.55\textwidth]{pres_pics/stack.png}\\
		$\Rightarrow$ Steckenbleiben der Personen
	\end{figure}
\end{frame}
\begin{frame}{Lösungsidee II}
	\begin{figure}
		\includegraphics[width=0.55\textwidth]{pres_pics/still_stack.png}\\
		$\Rightarrow$ Listenproblem von bereits verschwindenen Personen
	\end{figure}
\end{frame}
\begin{frame}{Verteilungsparameter}
	\begin{itemize}
		\item Bewegung von Beinen idealisiert als Pendelbewegung \cite{weidmann-1993}
		\begin{itemize}
			\item Obergrenze des natürlichen Gehens bei 2.0 $\frac{m}{s}$ \cite{weidmann-1993}
			\item Untergrenze des natürlichen Gehens bei 0.5 $\frac{m}{s}$ \cite{weidmann-1993}
		\end{itemize}
		\item Schrittlänge liegt bei ca. 0.63 $m$ $\Rightarrow$ 1.32 $\frac{m}{s}$
	\end{itemize}
\end{frame}
\begin{frame}{Verteilungsparameter}
	\begin{itemize}
		\item Bewegung von Beinen idealisiert als Pendelbewegung \cite{weidmann-1993}
		\begin{itemize}
			\item Obergrenze des natürlichen Gehens bei 2.0 $\frac{m}{s}$ \cite{weidmann-1993}
			\item Untergrenze des natürlichen Gehens bei 0.5 $\frac{m}{s}$ \cite{weidmann-1993}
		\end{itemize}
		\item Schrittlänge liegt bei ca. 0.63 $m$ $\Rightarrow$ 1.32 $\frac{m}{s}$ \cite{weidmann-1993}
	\end{itemize}
	\vspace{1cm}
	\hspace{1cm}$\Rightarrow$ Wahl der Parameter für Vadere: $0.9 \frac{m}{s}$\\\hspace{1.5cm}als mittlere Wunschgeschwindigkeit und Varianz $0.4$
\end{frame}
\begin{frame}{Dichte über JSON-Editor}
	\begin{itemize}
		\item Laden von .trajectories-Files
	\end{itemize}
\end{frame}
\begin{frame}{Dichte über JSON-Editor}
	\begin{itemize}
		\item Laden von .trajectories-Files
		\item Definition der Parameter\\(Startpunkt ($x_s$, $y_s$), Größe der Messfläche $A$ von Startpunkt, Größe der Obstacle-Fläche $A_O$, Pedestrian-Radius $r$, Anzahl der Zeitschritte $t_z$, Simulationsschrittlänge $t_{sim}$)
	\end{itemize}
\end{frame}
\begin{frame}{Dichte über JSON-Editor}
	\begin{itemize}
		\item Laden von .trajectories-Files
		\item Definition der Parameter\\(Startpunkt ($x_s$, $y_s$), Größe der Messfläche $A$ von Startpunkt, Größe der Obstacle-Fläche $A_O$, Pedestrian-Radius $r$, Anzahl der Zeitschritte $t$, Simulationsschrittlänge $t_{sim}$)
		\item Summiere die Fläche der Obstacles mit dem Mittelwert der Fläche der Pedestrians über die Zeitschritte $t_z$ durch die Gesamtfläche $A$
		\\\vspace{0.5cm}
		\centering
		$\rho_{A_t} = \frac{\frac{N_p \cdot r^2 \cdot \pi}{t_z \cdot t_{sim}}}{A - A_O}$\hspace{1cm}with $N_p$ number of all Pedestrians
	\end{itemize}
\end{frame}
\begin{frame}{Ausblick}
	\begin{figure}
		Verteilung der Spawnzeiten\\
		\includegraphics[width=\textwidth]{pres_pics/sirens.png}
	\end{figure}
\end{frame}
\begin{frame}{Ausblick}
	\begin{minipage}{0.50\textwidth}
		\includegraphics[width=\textwidth]{pres_pics/NAVI_TRAJ_miklosy.png}
	\end{minipage} \hfill
	\begin{minipage}{0.48\textwidth}
		\includegraphics[width=1.04\textwidth]{pres_pics/NAVI_TRAJ_feld.png}
	\end{minipage}
\end{frame}
